\documentclass[11pt]{article}
\usepackage[margin = 1in]{geometry}
\usepackage[amsfonts, amsmath, amssymb]{}
\usepackage[none]{hyphenat}
\usepackage{fancyhdr}
\usepackage{graphicx}
\usepackage{float}
\usepackage[nottoc, notlot, notlof]{tocbibind}

\pagestyle{fancy}
\fancyhead{} 
\fancyfoot{} % remove page number at the bottom
\fancyhead[L]{\slshape \MakeUppercase{Place Title Here}}
\fancyhead[R]{\slshape Student Name}
\fancyfoot[C]{\thepage}
% \renewcommand{\headrulewidth}{0pt} % remove horizontal line on top
\renewcommand{\footrulewidth}{0pt} % remove horizontal line on bottom (if there is)

% \parindent 0ex % remove indentation of paragraphs
% \setlength{\parindent}{4em}  % adjust length of indentation
% \setlength{\parskip}{1em}
\renewcommand{\baselinestretch}{1.5}

\begin{document}

\begin{titlepage}
\begin{center}
\vspace{1cm}
\Large{\textbf{IB Mathematics SL}}\\
\Large{\textbf{Internal Assessment}}\\
\vfill
\line(1,0){400}\\[1mm]
\huge{\textbf{This is a sample Title}}\\[3mm]
\Large{\textbf{-This is sample Sub Title-}}\\[1mm]
\line(1,0){400}\\
\vfill
By Student Name
Candidate \#\\
\today \\
\end{center}
\end{titlepage}

\tableofcontents
\thispagestyle{empty}
\clearpage
\setcounter{page}{1}

\section{Introduction}

There is a good chance that you have never written a paper in a math class before. So you
might be wondering why writing is required in your math class now.
The Greek word mathemas, from which we derive the word mathematics, embodies the
notions of knowledge, cognition, understanding, and perception. In the end, mathematics is
about ideas.\cite{DBHS2}\\

 In math classes at the university level, the ideas and concepts encountered are
more complex and sophisticated. The mathematics learned in college will include concepts
which cannot be expressed using just equations and formulas. Putting mathemas on paper
will require writing sentences and paragraphs in addition to the equations and formulas. \footnote{An example footnote}

\section{Scoring Criteria}

\subsection{Communication}
Mathematicians actually spend a great deal of time writing. If a mathematician wants to contribute to the greater body of mathematical knowledge, she must be able communicate her ideas in a way which is comprehensible to others. Thus, being able to write clearly is as important a mathematical skill as being able to solve equations.

Mastering the ability to write clear mathematical explanations is important for non-mathematicians as well. As you continue taking math courses in college, you will come to know more mathematics than most other people. When you use your mathematical knowledge in the future, you may be required to explain your thinking process to another person (like your boss, a co-worker, or an elected official), and it will be quite likely that this other person will know less math than you do. Learning how to communicate mathematical ideas clearly can help you advance in your career.

\subsection{Mathametical Presentation} 
For this same reason, just writing down your final conclusions in an assignment will not be sufficient for a university math class.

You should not confuse writing mathematics with “showing your work”. You will not be writing math papers to demonstrate that you have done the homework. Rather, you will be writing to demonstrate how well you understand mathematical ideas and concepts. A list of calculations without any context or explanation demonstrates that you’ve spent some time doing computations; however, a list of calculations without any explanations omits ideas. The ideas are the mathematics. So a page of computations without any writing or explanation contains no math. When you write a paper in a math class, your goal will be to communicate mathematical reasoning and ideas clearly to another person. The writing done in a math class is very similar to the writing done for other classes. Your are probably already used to writing papers in other subjects like psychology, history, and literature. You can follow many of the same guidelines in a mathematics paper as you would in a paper written about these other subjects.

\subsection{Personal Engagement}
Good writing observes the rules of grammar. This applies to writing in mathematics papers as well! When you write in a math class, you are expected to use correct grammar and spelling. Your writing should be clear and professional. Do not use any irregular abbreviations or shorthand forms which do not conform to standard writing conventions. 

Mathematics is written with sentences in paragraphs (And yes, paragraphs are important. It is not amusing to read a three-page paper consisting of just one paragraph.) There is however one element in mathematical writing which is not found in other types of writing: formulas. However, it may surprise you to know that in a math paper, formulas and equations follow the standard grammatical rules that apply to words.
\subsection{Reflection}
There are a couple of other important things to observe in the above example. Notice how “we” is used. The use of first person is common in mathematics, especially the plural “we”, so don’t be afraid to use the word “we” in the papers you write in your math class.

Another thing to notice is that important or long formulas are written on separate lines. You can make your mathematical writing easier to read if you place each important formula on a line of its own. It’s hard to pick out the important formulas below:

\subsection{Use Of Mathematics}
Use mathematical notation correctly. As you learn to write more complicated formulas, it is all too easy to leave out symbols from formulas. Learn how to use symbols properly!

\section{Conclusion}

\section{Using \LaTeX\ }

\pagebreak
\begin{thebibliography}{}

\bibitem{DBHS1}
Aloccer, Hood
``Diamond Bar High School."
\textit{International Assessment : Mathemetical Exploration},
Wed 27 2015


\bibitem{DBHS2}
Man, Hood
``Diamond Bar High School."
\textit{International Assessment : Mathemetical Exploration},
Wed 27 2015
\end{thebibliography}
\end{document}